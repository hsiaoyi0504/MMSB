%%%%%%%%%%%%%%%%%%%%%%%%%%%%%%%%%%%%%%%%%
% Short Sectioned Assignment
% LaTeX Template
% Version 1.0 (5/5/12)
%
% This template has been downloaded from:
% http://www.LaTeXTemplates.com
%
% Original author:
% Frits Wenneker (http://www.howtotex.com)
%
% License:
% CC BY-NC-SA 3.0 (http://creativecommons.org/licenses/by-nc-sa/3.0/)
%
%%%%%%%%%%%%%%%%%%%%%%%%%%%%%%%%%%%%%%%%%

%----------------------------------------------------------------------------------------
%	PACKAGES AND OTHER DOCUMENT CONFIGURATIONS
%----------------------------------------------------------------------------------------

\documentclass[paper=a4, fontsize=11pt]{scrartcl} % A4 paper and 11pt font size

\usepackage[T1]{fontenc} % Use 8-bit encoding that has 256 glyphs
\usepackage{fourier} % Use the Adobe Utopia font for the document - comment this line to return to the LaTeX default
\usepackage[english]{babel} % English language/hyphenation
\usepackage{amsmath,amsfonts,amsthm} % Math packages
\usepackage{enumerate}
\usepackage{lipsum} % Used for inserting dummy 'Lorem ipsum' text into the template
\usepackage{sectsty} % Allows customizing section commands
\allsectionsfont{\centering \normalfont\scshape} % Make all sections centered, the default font and small caps
\sectionfont{\raggedright}
\usepackage[english]{babel}
\usepackage{mathtools}
\usepackage{graphicx}
\graphicspath{ {img/} }
\DeclareGraphicsExtensions{.png,.jpg}
\usepackage{color}
\usepackage{pbox}

\usepackage{fancyhdr} % Custom headers and footers
\pagestyle{fancyplain} % Makes all pages in the document conform to the custom headers and footers
\fancyhead{} % No page header - if you want one, create it in the same way as the footers below
\fancyfoot[L]{} % Empty left footer
\fancyfoot[C]{} % Empty center footer
\fancyfoot[R]{\thepage} % Page numbering for right footer
\renewcommand{\headrulewidth}{0pt} % Remove header underlines
\renewcommand{\footrulewidth}{0pt} % Remove footer underlines
\setlength{\headheight}{13.6pt} % Customize the height of the header

\numberwithin{equation}{section} % Number equations within sections (i.e. 1.1, 1.2, 2.1, 2.2 instead of 1, 2, 3, 4)
\numberwithin{figure}{section} % Number figures within sections (i.e. 1.1, 1.2, 2.1, 2.2 instead of 1, 2, 3, 4)
\numberwithin{table}{section} % Number tables within sections (i.e. 1.1, 1.2, 2.1, 2.2 instead of 1, 2, 3, 4)

\setlength\parindent{0pt} % Removes all indentation from paragraphs - comment this line for an assignment with lots of text

%----------------------------------------------------------------------------------------
%	TITLE SECTION
%----------------------------------------------------------------------------------------

\newcommand{\horrule}[1]{\rule{\linewidth}{#1}} % Create horizontal rule command with 1 argument of height

\title{	
\normalfont \normalsize 
\textsc{National Taiwan University, \\ Graduate Institute of Biomedical Engineering and Bioinformatics} \\ [25pt] % Your university, school and/or department name(s)
\horrule{0.5pt} \\[0.4cm] % Thin top horizontal rule
\huge BEBI5009:\\Mathematical Modeling of System Biology \\ Homework 5 \\ % The assignment title
\horrule{2pt} \\[0.5cm] % Thick bottom horizontal rule
}

\author{Yi Hsiao\\R04945027} % Your name

\date{\normalsize\today} % Today's date or a custom date

\begin{document}

\maketitle % Print the title

\newpage
\section{5.6.3 Metabolic Control Analysis: supply and demand}
	Consider the two-step reaction chain $\xrightarrow{v_0}S\xrightarrow{v_1}$, where the reactions are catalysed by enzymes $E_0$ and $E_1$ with concentrations $e_0$ and $e_1$. The Summation Theorem (Section 5.2.1) states that
	\begin{align*}
		C_{e_0}^J+C_{e_1}^J=1
	\end{align*}
	A complementary result, the Connectivity Theorem (Heinrich and Schuster, 1996) states that
	\begin{align*}
		C_{e_0}^J\epsilon_S^0+C_{e_1}^J\epsilon_S^1=0
	\end{align*}
	\begin{enumerate}[a)]
		\item Use these two statements to determine the flux control coefficients of the two reactions as
		\begin{align*}
			C_{e_0}^J=\frac{\epsilon_S^1}{\epsilon_S^1-\epsilon_S^0}\\
			C_{e_1}^J=\frac{-\epsilon_S^0}{\epsilon_S^1-\epsilon_S^0}
		\end{align*}

		According to Cramer's Rule, it is easy to solve these two variables equations:
		\begin{align*}
			C_{e_0}^J=\frac{
				\begin{vmatrix}
					1 & 1  \\
   					0 & \epsilon_S^1
				\end{vmatrix}
			}{
				\begin{vmatrix}
					1 & 1  \\
   					\epsilon_S^0 & \epsilon_S^1
				\end{vmatrix}
			}=\frac{\epsilon_S^1}{\epsilon_S^1-\epsilon_S^0}\\
			C_{e_1}^J=\frac{
				\begin{vmatrix}
					1 & 1  \\
   					\epsilon_S^0 & 0
				\end{vmatrix}
			}{
				\begin{vmatrix}
					1 & 1  \\
   					\epsilon_S^0 & \epsilon_S^1
				\end{vmatrix}
			}=\frac{-\epsilon_S^0}{\epsilon_S^1-\epsilon_S^0}
		\end{align*}

		\item In addressing the control of flux through the pathway, we can think of $v_0$ as the 'supply rate' and $v_1$ as the 'demand rate'. Given the result in part (a), under what conditions on the elasticities $\epsilon_S^0$ and $\epsilon_S^1$ will a perturbation in the rate of supply affect pathway flux more than an equivalent perturbation in the rate of demand?

		When a perturbation in the rate of supply affect pathway flux more than an equivalent perturbation in the rate of demand, $C_{e_0}^J>C_{e_0}^J$, so
		\begin{align*}
		&\frac{\epsilon_S^1}{\epsilon_S^1-\epsilon_S^0} > \frac{-\epsilon_S^0}{\epsilon_S^1-\epsilon_S^0} \\
		\Rightarrow &\epsilon_S^1 >-\epsilon_S^0
		\end{align*}

		\item Suppose the rate laws are given as $v_0 = e_0(k_0X - k_{-1}[S])$ and $v_1 = e_1k_1[S]$, where X is the constant concentration of the pathway substrate. Verify that the elasticities are
		\begin{align*}
			\epsilon_S^0 &= \frac{-k_{-1}[S]}{k_0X-k_{-1}[S]}\\
			and \quad \epsilon_S^1 &= 1
		\end{align*}
		Determine conditions on the parameters under which perturbation in the supply reaction $v_0$ will have a more significant effect than perturbation in the demand reaction $v_1$. Hint: at steady state $k_0X-k_{-1}s = e_1k_1s/e_0$.

		According to the definition of the elasticity,
		\begin{align*}
			\epsilon_S^0 &= \frac{[S]}{v_0}\frac{\partial v_0}{\partial[S]}=\frac{[S]}{e_0(k_0X - k_{-1}[S])}(-e_0k_{-1})=\frac{-k_{-1}[S]}{k_0X - k_{-1}[S]} \\
			\epsilon_S^1 &= \frac{[S]}{v_1}\frac{\partial v_1}{\partial[S]}=\frac{[S]}{e_1k_1[S]}e_1k_1=1
		\end{align*}
		From the result of b), to get a perturbation in the supplyreaction that has a more significant effect than perturbation in the demand reaction, we need $\epsilon_S^1 >-\epsilon_S^0$, so
		\begin{align*}
			& \epsilon_S^1 = 1 > -\frac{-k_{-1}[S]}{k_0X-k_{-1}[S]} = -\epsilon_S^0\\
			& \Rightarrow k_0X-k_{-1}[S] > k_{-1}[S] \\
			& \Rightarrow k_0X > 2k_{-1}[S] 
		\end{align*}
	\end{enumerate}

\section{6.8.18 Frequency response analysis of a two-component signaling pathway}
	\begin{enumerate}[a)]
		\item Following the procedure in Section 6.6.3, determine the linearization of the two-component signaling pathway model of Section 6.1.1 at an arbitrary nominal input value. Use species conservations to reduce the model before linearizing.

		The model equations are
		\begin{align*}
			\frac{d[R](t)}{dt}  &= -k_1[R](t)[L](t) + k_{-1}[RL](t)\\
			\frac{d[RL](t)}{dt} &= k_1[R](t)[L](t) - k_{-1}[RL](t)\\
			\frac{d[P](t)}{dt}  &= -k_2[P](t)[RL](t) + k_3[P^*](t)\\
			\frac{d[P^*](t)}{dt}&= k_2[P](t)[RL](t) - k_3[P^*](t)
		\end{align*}
		Species conservationsa are
		\begin{align*}
			\frac{d[R](t)}{dt}  + \frac{d[RL](t)}{dt} = 0\\
			\frac{d[P](t)}{dt}  + \frac{d[P^*](t)}{dt} = 0
		\end{align*}
		Let $[R](t)+[RL](t)=R_T$, $[P](t)+[P^*](t)=P_T$. Then, the model equations can be reduced to two differential equations shwon below
		\begin{align*}
			\frac{d[RL](t)}{dt} &= k_1(R_T-[RL](t))[L](t) - k_{-1}[RL](t)\\
			\frac{d[P^*](t)}{dt}&= k_2(P_T-[P^*](t))[RL](t) - k_3[P^*](t)
		\end{align*}
		Here, the input signal is the ligand level [L], while the output is the concentration of active $[P^*]$. Comparing with equation (6.3), with $\mathbf{x} = ([RL],[P^*])$ and u = [L], this system defines $\mathbf{f}$ as a vector of function with two coefficients:
		\begin{align*}
			f_1([RL],[P^*])= k_1(R_T-[RL](t))[L](t) - k_{-1}[RL](t)\\
			f_2([RL],[P^*])= k_2(P_T-[P^*](t))[RL](t) - k_3[P^*](t)
		\end{align*}
		The output function(equation(6.4)) takes the simple form
		\begin{align*}
			h([RL],[P^*])=[P^*]
		\end{align*}
		The linearization takes the form of a linear input-output system:
		\begin{align*}
			\frac{d\mathbf{x}(t)}{dt} &= \mathbf{Ax}(t)+\mathbf{B}u(t)\\
			y(t) &= \mathbf{Cx}(t) + Du(t) 
		\end{align*}
		where the Jacobian
		\begin{align*}
			\mathbf{A}=
			\begin{bmatrix}
					\frac{\partial f_1}{\partial [RL]}&\frac{\partial f_1}{\partial [P^*]}\\
					\frac{\partial f_2}{\partial [RL]}&\frac{\partial f_2}{\partial [P^*]}
			\end{bmatrix}
			=
			\begin{bmatrix}
					-k_1[L]-k_{-1} & 0\\
					k_2(P_T-[P*]) & -k_2[RL]-k_3
			\end{bmatrix},
		\end{align*}
		the linearized input map
		\begin{align*}
			\mathbf{B}=
			\begin{bmatrix}
					\frac{\partial f_1}{\partial [L]}\\
					\frac{\partial f_2}{\partial [L]}
			\end{bmatrix}
			=
			\begin{bmatrix}
					k_1(R_T-[RL] \\
					0
			\end{bmatrix},
		\end{align*}
		the linearized output map is
		\begin{align*}
			\mathbf{C}=
			\begin{bmatrix}
					\frac{\partial h}{\partial [RL]} & \frac{\partial h}{\partial [P^*]}
			\end{bmatrix}
			=
			\begin{bmatrix}
					0 & 1
			\end{bmatrix},
		\end{align*}
		and no feed-through:
		\begin{align*}
			D = \frac{\partial h}{\partial [L]} = 0
		\end{align*}

		\item Simulate the model to determine the steady state corresponding to a nominal input of $L_T = 0.04$. Use MATLAB to generate the magnitude Bode plot of the corresponding frequency response (details in Appendix C).

		Since $H(\omega)= \mathbf{C}(i\omega\mathbf{I-A})^{-1}\mathbf{B}+D$, based on the result of a), we can derive that
		\begin{align*}
			H(\omega)= 
			\begin{bmatrix}
				0 & 1
			\end{bmatrix}
			(
			\begin{bmatrix}
				i\omega & 0\\
				0 & i\omega
			\end{bmatrix}
			-
			\begin{bmatrix}
				-k_1[L]-k_{-1} & 0\\
				k_2(P_T-[P*]) & -k_2[RL]-k_3
			\end{bmatrix})^{-1}
			\begin{bmatrix}
				k_1(R_T-[RL] \\0
			\end{bmatrix}
			+ 0
		\end{align*}
		Let $k_1=5$, $k_{-1}=1$, $k_2=6$, $k_3=2$, $R_T=2$, and $P_2=8$ (initially, [RL]=0 and [$P^*$]=0). When [L] = $L_T$ = 0.04,
		\begin{align*}
			H(\omega)&= 
			\begin{bmatrix}
				0 & 1
			\end{bmatrix}
			(
			\begin{bmatrix}
				i\omega & 0\\
				0 & i\omega
			\end{bmatrix}
			-
			\begin{bmatrix}
				-1.2 & 0\\
				48 & -2
			\end{bmatrix})^{-1}
			\begin{bmatrix}
				10 \\0
			\end{bmatrix}
			+ 0
			=
			\begin{bmatrix}
				0 & 1
			\end{bmatrix}
			(
			\begin{bmatrix}
				i\omega+1.2 & 0\\
				-48 & i\omega+2
			\end{bmatrix}
			)^{-1}
			\begin{bmatrix}
				10 \\0
			\end{bmatrix}
			\\&=
			\begin{bmatrix}
				0 & 1
			\end{bmatrix}
			\frac{1}{-\omega^2+3.2i\omega+2.4}
			\begin{bmatrix}
				i\omega+2 & 0\\
				48 & i\omega+1.2
			\end{bmatrix}
			\begin{bmatrix}
				10 \\0
			\end{bmatrix}
			= \frac{480}{-\omega^2+3.2i\omega+2.4}
		\end{align*}
		Then, we can also obtain the gain of the system:
		\begin{align*}
			Gain(\omega)=\frac{480}{\sqrt{(2.4-\omega^2)^2+10.24\omega^2}}
		\end{align*}
		Low frequency gain is then
		\begin{align*}
			Gain(0)=\frac{480}{\sqrt{(2.4)^2}}=480/2.4=200=46.02dB
		\end{align*}
		The MATLAB simulation result is shown below. It also confirms the derivation above.\\
		\includegraphics[scale=0.3]{{2.b}.jpg}

		\item  Repeat part (b) for a nominal input of of $L_T = 0.4$. Use Figure 6.3 to explain the difference in the frequency response at these two nominal input values.

		Similar to the procedure of b), we can get
		\begin{align*}
			H(\omega)&= 
			\begin{bmatrix}
				0 & 1
			\end{bmatrix}
			(
			\begin{bmatrix}
				i\omega & 0\\
				0 & i\omega
			\end{bmatrix}
			-
			\begin{bmatrix}
				-3 & 0\\
				48 & -2
			\end{bmatrix})^{-1}
			\begin{bmatrix}
				10 \\0
			\end{bmatrix}
			+ 0
			=
			\begin{bmatrix}
				0 & 1
			\end{bmatrix}
			(
			\begin{bmatrix}
				i\omega+3 & 0\\
				-48 & i\omega+2
			\end{bmatrix}
			)^{-1}
			\begin{bmatrix}
				10 \\0
			\end{bmatrix}
			\\&=
			\begin{bmatrix}
				0 & 1
			\end{bmatrix}
			\frac{1}{-\omega^2+5i\omega+6}
			\begin{bmatrix}
				i\omega+2 & 0\\
				48 & i\omega+3
			\end{bmatrix}
			\begin{bmatrix}
				10 \\0
			\end{bmatrix}
			= \frac{480}{-\omega^2+5i\omega+6}
		\end{align*}
		Then, we can also obtain the gain of the system:
		\begin{align*}
			Gain(\omega)=\frac{480}{\sqrt{(6-\omega^2)^2+25\omega^2}}
		\end{align*}
		Low frequency gain is then
		\begin{align*}
			Gain(0)=\frac{480}{\sqrt{(6)^2}}=480/6=80=38.06dB
		\end{align*}
		The MATLAB simulation result is shown below. It also confirms the derivation above.\\
		\includegraphics[scale=0.3]{{2.c}.jpg}

		The shape of these two bode plot are similar. However, the bode plot of lower $L_T$ shows larger magnitude of gain than higher $L_T$. This behaviror can be explained by figure 6.3. As you can see in figure 6.3.B, lower total ligand concentration results in higher ratio for response(active response protein) to input signals (slope is larger when the total ligand concentration is lower).

	\end{enumerate}
	Appendix C- Frequency response analysis \\
	As discussed in Section 6.6, a linear input-output system can be specified as a set of four matrices: A,B, C, and D. Once these are specified in MATLAB, the frequency response can be easily generated and the corresponding Bode plots can be produced. The command [num den]=ss2tf(A,B,C,D) determines the transfer function for the system, specified in terms of the coefficients of the numerator and denominator polynomials. A MATLAB transfer function can then be created with the command sys=tf(num,den), from which Bode plots can be produced with bode(sys).
\end{document}