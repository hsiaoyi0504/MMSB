%%%%%%%%%%%%%%%%%%%%%%%%%%%%%%%%%%%%%%%%%
% Short Sectioned Assignment
% LaTeX Template
% Version 1.0 (5/5/12)
%
% This template has been downloaded from:
% http://www.LaTeXTemplates.com
%
% Original author:
% Frits Wenneker (http://www.howtotex.com)
%
% License:
% CC BY-NC-SA 3.0 (http://creativecommons.org/licenses/by-nc-sa/3.0/)
%
%%%%%%%%%%%%%%%%%%%%%%%%%%%%%%%%%%%%%%%%%

%----------------------------------------------------------------------------------------
%	PACKAGES AND OTHER DOCUMENT CONFIGURATIONS
%----------------------------------------------------------------------------------------

\documentclass[paper=a4, fontsize=11pt]{scrartcl} % A4 paper and 11pt font size

\usepackage[T1]{fontenc} % Use 8-bit encoding that has 256 glyphs
\usepackage{fourier} % Use the Adobe Utopia font for the document - comment this line to return to the LaTeX default
\usepackage[english]{babel} % English language/hyphenation
\usepackage{amsmath,amsfonts,amsthm} % Math packages
\usepackage{enumerate}
\usepackage{lipsum} % Used for inserting dummy 'Lorem ipsum' text into the template
\usepackage{sectsty} % Allows customizing section commands
\allsectionsfont{\centering \normalfont\scshape} % Make all sections centered, the default font and small caps
\sectionfont{\raggedright}
\usepackage[english]{babel}
\usepackage{mathtools}
\usepackage{graphicx}
\graphicspath{ {img/} }
\DeclareGraphicsExtensions{.png,.jpg}
\usepackage{color}
\usepackage{pbox}

\usepackage{fancyhdr} % Custom headers and footers
\pagestyle{fancyplain} % Makes all pages in the document conform to the custom headers and footers
\fancyhead{} % No page header - if you want one, create it in the same way as the footers below
\fancyfoot[L]{} % Empty left footer
\fancyfoot[C]{} % Empty center footer
\fancyfoot[R]{\thepage} % Page numbering for right footer
\renewcommand{\headrulewidth}{0pt} % Remove header underlines
\renewcommand{\footrulewidth}{0pt} % Remove footer underlines
\setlength{\headheight}{13.6pt} % Customize the height of the header

\numberwithin{equation}{section} % Number equations within sections (i.e. 1.1, 1.2, 2.1, 2.2 instead of 1, 2, 3, 4)
\numberwithin{figure}{section} % Number figures within sections (i.e. 1.1, 1.2, 2.1, 2.2 instead of 1, 2, 3, 4)
\numberwithin{table}{section} % Number tables within sections (i.e. 1.1, 1.2, 2.1, 2.2 instead of 1, 2, 3, 4)

\setlength\parindent{0pt} % Removes all indentation from paragraphs - comment this line for an assignment with lots of text

%----------------------------------------------------------------------------------------
%	TITLE SECTION
%----------------------------------------------------------------------------------------

\newcommand{\horrule}[1]{\rule{\linewidth}{#1}} % Create horizontal rule command with 1 argument of height

\title{	
\normalfont \normalsize 
\textsc{National Taiwan University, \\ Graduate Institute of Biomedical Engineering and Bioinformatics} \\ [25pt] % Your university, school and/or department name(s)
\horrule{0.5pt} \\[0.4cm] % Thin top horizontal rule
\huge BEBI5009:\\Mathematical Modeling of System Biology \\ Homework 5 \\ % The assignment title
\horrule{2pt} \\[0.5cm] % Thick bottom horizontal rule
}

\author{Yi Hsiao\\R04945027} % Your name

\date{\normalsize\today} % Today's date or a custom date

\begin{document}

\maketitle % Print the title

\newpage
\section{5.6.3 Metabolic Control Analysis: supply and demand}
	Consider the two-step reaction chain $\xrightarrow{v_0}S\xrightarrow{v_1}$, where the reactions are catalysed by enzymes $E_0$ and $E_1$ with concentrations $e_0$ and $e_1$. The Summation Theorem (Section 5.2.1) states that
	\begin{align*}
		C_{e_0}^J+C_{e_1}^J=1
	\end{align*}
	A complementary result, the Connectivity Theorem (Heinrich and Schuster, 1996) states that
	\begin{align*}
		C_{e_0}^J\epsilon_S^0+C_{e_1}^J\epsilon_S^1=0
	\end{align*}
	\begin{enumerate}[a)]
		\item Use these two statements to determine the flux control coefficients of the two reactions as
		\begin{align*}
			C_{e_0}^J=\frac{\epsilon_S^1}{\epsilon_S^1-\epsilon_S^0}\\
			C_{e_1}^J=\frac{-\epsilon_S^0}{\epsilon_S^1-\epsilon_S^0}
		\end{align*}

		According to Cramer's Rule, it is easy to solve these two variables equations:
		\begin{align*}
			C_{e_0}^J=\frac{
				\begin{vmatrix}
					1 & 1  \\
   					0 & \epsilon_S^1
				\end{vmatrix}
			}{
				\begin{vmatrix}
					1 & 1  \\
   					\epsilon_S^0 & \epsilon_S^1
				\end{vmatrix}
			}=\frac{\epsilon_S^1}{\epsilon_S^1-\epsilon_S^0}\\
			C_{e_1}^J=\frac{
				\begin{vmatrix}
					1 & 1  \\
   					\epsilon_S^0 & 0
				\end{vmatrix}
			}{
				\begin{vmatrix}
					1 & 1  \\
   					\epsilon_S^0 & \epsilon_S^1
				\end{vmatrix}
			}=\frac{-\epsilon_S^0}{\epsilon_S^1-\epsilon_S^0}
		\end{align*}

		\item In addressing the control of flux through the pathway, we can think of $v_0$ as the 'supply rate' and $v_1$ as the 'demand rate'. Given the result in part (a), under what conditions on the elasticities $\epsilon_S^0$ and $\epsilon_S^1$ will a perturbation in the rate of supply affect pathway flux more than an equivalent perturbation in the rate of demand?



		\item Suppose the rate laws are given as $v_0 = e_0(k_0X - k_{-1}[S])$ and $v_1 = e_1k_1[S]$, where X is the constant concentration of the pathway substrate. Verify that the elasticities are
		\begin{align*}
			\epsilon_S^0 &= \frac{k_{-1}[S]}{k_0X-k_{-1}[S]}\\
			and \quad \epsilon_S^1 &= 1
		\end{align*}
		Determine conditions on the parameters under which perturbation in the supply reaction $v_0$ will have a more significant effect than perturbation in the demand reaction $v_1$. Hint: at steady state $k_0X-k_{-1}s = e_1k_1s/e_0$.



	\end{enumerate}

\section{6.8.18 Frequency response analysis of a two-component signaling pathway}
	\begin{enumerate}[a)]
		\item Following the procedure in Section 6.6.3, determine the linearization of the two-component signaling pathway model of Section 6.1.1 at an arbitrary nominal input value. Use species conservations to reduce the model before linearizing.

		\item example q2
	\end{enumerate}
\end{document}