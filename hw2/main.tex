%%%%%%%%%%%%%%%%%%%%%%%%%%%%%%%%%%%%%%%%%
% Short Sectioned Assignment
% LaTeX Template
% Version 1.0 (5/5/12)
%
% This template has been downloaded from:
% http://www.LaTeXTemplates.com
%
% Original author:
% Frits Wenneker (http://www.howtotex.com)
%
% License:
% CC BY-NC-SA 3.0 (http://creativecommons.org/licenses/by-nc-sa/3.0/)
%
%%%%%%%%%%%%%%%%%%%%%%%%%%%%%%%%%%%%%%%%%

%----------------------------------------------------------------------------------------
%	PACKAGES AND OTHER DOCUMENT CONFIGURATIONS
%----------------------------------------------------------------------------------------

\documentclass[paper=a4, fontsize=11pt]{scrartcl} % A4 paper and 11pt font size

\usepackage[T1]{fontenc} % Use 8-bit encoding that has 256 glyphs
\usepackage{fourier} % Use the Adobe Utopia font for the document - comment this line to return to the LaTeX default
\usepackage[english]{babel} % English language/hyphenation
\usepackage{enumerate}
\usepackage{amsmath,amsfonts,amsthm} % Math packages
\usepackage{lipsum} % Used for inserting dummy 'Lorem ipsum' text into the template
\usepackage{sectsty} % Allows customizing section commands
\allsectionsfont{\centering \normalfont\scshape} % Make all sections centered, the default font and small caps
\sectionfont{\raggedright}
\usepackage[english]{babel}
\usepackage{mathtools}
\usepackage{graphicx}
\graphicspath{ {img/} }
\DeclareGraphicsExtensions{.png,.jpg}
\usepackage{color}
\usepackage{pbox}

\usepackage{fancyhdr} % Custom headers and footers
\pagestyle{fancyplain} % Makes all pages in the document conform to the custom headers and footers
\fancyhead{} % No page header - if you want one, create it in the same way as the footers below
\fancyfoot[L]{} % Empty left footer
\fancyfoot[C]{} % Empty center footer
\fancyfoot[R]{\thepage} % Page numbering for right footer
\renewcommand{\headrulewidth}{0pt} % Remove header underlines
\renewcommand{\footrulewidth}{0pt} % Remove footer underlines
\setlength{\headheight}{13.6pt} % Customize the height of the header

\numberwithin{equation}{section} % Number equations within sections (i.e. 1.1, 1.2, 2.1, 2.2 instead of 1, 2, 3, 4)
\numberwithin{figure}{section} % Number figures within sections (i.e. 1.1, 1.2, 2.1, 2.2 instead of 1, 2, 3, 4)
\numberwithin{table}{section} % Number tables within sections (i.e. 1.1, 1.2, 2.1, 2.2 instead of 1, 2, 3, 4)

\setlength\parindent{0pt} % Removes all indentation from paragraphs - comment this line for an assignment with lots of text

%----------------------------------------------------------------------------------------
%	TITLE SECTION
%----------------------------------------------------------------------------------------

\newcommand{\horrule}[1]{\rule{\linewidth}{#1}} % Create horizontal rule command with 1 argument of height

\title{	
\normalfont \normalsize 
\textsc{National Taiwan University, \\ Graduate Institute of Biomedical Engineering and Bioinformatics} \\ [25pt] % Your university, school and/or department name(s)
\horrule{0.5pt} \\[0.4cm] % Thin top horizontal rule
\huge BEBI5009:\\Mathematical Modeling of System Biology \\ Homework 2 \\ % The assignment title
\horrule{2pt} \\[0.5cm] % Thick bottom horizontal rule
}

\author{Yi Hsiao} % Your name

\date{\normalsize\today} % Today's date or a custom date

\begin{document}

\maketitle % Print the title

\newpage
\section{3.7.5 Michaelis-Menten kinetics:first-order approximation.}
	Consider the reaction chain
	\begin{gather*}
		\xrightarrow{v_0} S_1 \xrightarrow{v_1} S_2 \xrightarrow{v_2} S_3 \xrightarrow{v_3} 
	\end{gather*}
	in which the $v_i$ are labels for the reaction rates (not mass-action constants). Take the rate $v_0$ as fixed and presume the other reactions follow Michaelis-Menten kinetics, with
	\begin{gather*}
		v_i=\frac{V^i_{max}s_i}{K_{Mi}+s_i},
	\end{gather*}
	where $s_i=[S_i]$. Take parameter values (in mM/min) $v_0=2$, $V^1_{max} = 9$, $V^2_{max}=12$, $V^3_{max}=15$;(in mM) $K_{M1} = 1$, $K_{M2} = 0.4$, $K_{M3} = 3$.
	\begin{enumerate}[a)]
		\item Simulate the system from initial conditions (in mM) ($s_1$, $s_2$, $s_3$) = (0.3, 0.2, 0.1). Repeat with initial condition ($s_1$, $s_2$, $s_3$) = (6, 4, 4).
		
		According to the question, we can easily derive following equations:
		\begin{gather*}
			\frac{dS_i}{dt}=v_i-v_{i-1}=
			\begin{cases}
				\frac{V^i_{max}s_i}{K_Mi+s_i}-v_0  \quad if \quad i = 1\\
				\frac{V^i_{max}s_i}{K_Mi+s_i}-\frac{V^{i-1}_{max}s_{i-1}}{K_{Mi-1}+s_{i-1}} \quad if \quad i = 2,3\\
			\end{cases}
		\end{gather*}
		Thus, we can simulate the system with these equations. The simulation result are shown below.
		\begin{enumerate}[i)]
			\item Simulation result with initial condition (in mM) ($s_1$, $s_2$, $s_3$) = (0.3, 0.2, 0.1).

			\item Simulation result with initial condition (in mM) ($s_1$, $s_2$, $s_3$) = (6, 4, 4).

		\end{enumerate}

		\item Generate an approximate model in which the rates of reactions 1, 2, and 3 follow first-order mass-action kinetics (i.e. $v_i = k_is_i$, for i = 1, 2, 3). Choose values for the rate constants $k_i$ that give a good approximation to the original nonlinear model. Explain your reasoning. (Hint: Exercise 3.1.2(b) provides one viable approach.)

		For $s$ is small, the reaction rate of Michaelis-Menten kinetics can be approximated:
		\begin{gather*}
			v_i=\frac{V^i_{max}s_i}{K_{Mi}+s_i}\approx \frac{V^i_{max}s_i}{K_{Mi}}
		\end{gather*}
		Then, it is intuitive to choose rate constants $k_i=\frac{V^i_{max}}{K_{Mi}}$

		\item Simulate your simpler (mass-action based) model from the sets of initial conditions in part (a). Comment on the fit. If the approximation is better in one case than the other, explain why.


	\end{enumerate}
\section{3.7.8 Allosteric activation}
	Consider an allosteric activation scheme in which an allosteric activator must be bound before an enzyme can bind substrate. This is called compulsory activation. The reaction scheme resembles a two-substrate reaction, but the enzyme-activator complex stays intact after the product dissociates:
	\begin{gather*}
		R + E \xrightleftharpoons[k_{-1}]{k_1} ER\\
		ER + S \xrightleftharpoons[k_{-2}]{k_2} ERS \xrightarrow{k_3} P + ER,
	\end{gather*}
	where R is the allosteric activator (regulator).
	\begin{enumerate}[a)]
		\item  Apply a quasi-steady-state assumption to the two complexes ER and ERS (and use enzyme conservation) to verify that the rate law takes the form
		\begin{gather*}
			v=\frac{srk_3e_T}{r\frac{k_{-2}+k_3}{k_2}+\frac{k_{-1}(k_{-2}+k_3)}{k_1k_2}+sr}=\frac{V_{max}sr}{K_1r+K_2+rs},
		\end{gather*}
		where r is the regulator concentration and s is the substrate concentration

		First, we can write down the differential equations for all species:
		\begin{align*}
			&\frac{d[R]}{dt}=-k_1[R][E]+k_{-1}[ER]\\
			&\frac{d[E]}{dt}=-k_1[R][E]+k_{-1}[ER]\\
			&\frac{d[ER]}{dt}=k_1[R][E]-k_{-1}[ER]-k_2[ER][S]+k_{-2}[ERS]+k_3[ERS]\\
			&\frac{d[S]}{dt}=-k_2[ER][S]+k_{-2}[ERS]\\
			&\frac{d[ERS]}{dt}=k_2[ER][S]-k_{-2}[ERS]-k_3[ERS]\\
			&\frac{d[P]}{dt}=k_3[ERS]=v
		\end{align*}
		By applying a quasi-steady-state assumption to the two complexes ER and ESR, we know that:
		\begin{align*}
			&0=k_2[ER]_{ss}[S]-k_{-2}[ERS]_{ss}-k_3[ERS]_{ss}\\
			\Rightarrow &[ER]_{ss}=\frac{k_{-2}+k_3}{k_2[S]}[ERS]_{ss}\\
			&0=k_1[R][E]-k_{-1}[ER]_{ss}-k_2[ER]_{ss}[S]+k_{-2}[ERS]_{ss}+k_3[ERS]_{ss}\\
			\Rightarrow&0=k_1[R](e_T-\frac{k_{-2}+k_3}{k_2[S]}[ERS]_{ss}-[ERS]_{ss})-k_{-1}\frac{k_{-2}+k_3}{k_2[S]}[ERS]_{ss}-k_2\frac{k_{-2}+k_3}{k_2[S]}[ERS]_{ss}[S]\\
			&+k_{-2}[ERS]_{ss}+k_3[ERS]_{ss}\\
			\Rightarrow& (k_1[R]\frac{k_{-2}+k_3}{k_2[S]}+k_1[R]+k_{-1}\frac{k_{-2}+k_3}{k_2[S]})[ERS]_{ss}=k_1[R]e_T\\
			\Rightarrow& [ERS]_{ss}= \frac{e_T[R][S]}{[R]\frac{k_{-2}+k_3}{k_2}+[R][S]+\frac{k_{-1}(k_{-2}+k_3)}{k_1k_2}} = \frac{e_T[S][R]}{[R]\frac{k_{-2}+k_3}{k_2}+\frac{k_{-1}(k_{-2}+k_3)}{k_1k_2}+[R][S]}\\
			\Rightarrow& v=k_3[ERS]=\frac{k_3e_T[S][R]}{\frac{k_{-2}+k_3}{k_2}[R]+\frac{k_{-1}(k_{-2}+k_3)}{k_1k_2}+[R][S]}
		\end{align*}
		where $e_t$ is from the conservation of enzyme:
		\begin{align*}
			e_T=[E]+[ER]_{ss}+[ERS]_{ss}
		\end{align*}
		Further substitute the equation with $K_1=\frac{k_{-2}+k_3}{k_2}$, $K_2=\frac{k_{-1}(k_{-2}+k_3)}{k_1k_2}$, $V_{max}=k_3e_T$. Then, we get the same result which is specified in the question. That is,
		\begin{align*}
			v=\frac{k_3e_T[S][R]}{\frac{k_{-2}+k_3}{k_2}[R]+\frac{k_{-1}(k_{-2}+k_3)}{k_1k_2}+[R][S]}=\frac{V_{max}[S][R]}{K_1[R]+K2+[R][S]}
		\end{align*}

		\item Next, consider the case in which catalysis can only occur after $n$ regulator molecules have bound. Assuming the the binding involves strong cooperativity, we can approximate the regulator-binding events by:
		\begin{gather*}
			nR + E \xrightleftharpoons[k_{-1}]{k_1} ER_n
		\end{gather*}
		Verify that in this case the rate law takes the form
		\begin{gather*}
			v=\frac{V_{max}sr^n}{K_1r^n+K_2+r^ns}.
		\end{gather*}

		The procedures are really similar to those in part (a).
		First, we can write down the differential equations for all species:
		\begin{align*}
			&\frac{d[R]}{dt}=-k_1[R]^n[E]+k_{-1}[ER_n]\\
			&\frac{d[E]}{dt}=-k_1[R]^n[E]+k_{-1}[ER_n]\\
			&\frac{d[ER_n]}{dt}=k_1[R]^n[E]-k_{-1}[ER_n]-k_2[ER_n][S]+k_{-2}[ER_nS]+k_3[ER_nS]\\
			&\frac{d[S]}{dt}=-k_2[ER_n][S]+k_{-2}[ER_nS]\\
			&\frac{d[ER_nS]}{dt}=k_2[ER_n][S]-k_{-2}[ER_nS]-k_3[ER_nS]\\
			&\frac{d[P]}{dt}=k_3[ER_nS]=v
		\end{align*}
		If we use $[R']=[R]^n$, $[ER']=[ER_n]$, $[ERS']=[ER_nS]$ to represent the equations above, we will get the same set equations with $[R']$, $[ER']$, and $[ERS']$ replacing $[R]$, $[ER]$, and $[ERS]$, respectively. So, we can similarly conclude that we get the same result which is specified in the question. That is,
		\begin{align*}
			v=\frac{k_3e_T[S][R']}{\frac{k_{-2}+k_3}{k_2}[R']+\frac{k_{-1}(k_{-2}+k_3)}{k_1k_2}+[R'][S]}=\frac{V_{max}[S][R']}{K_1[R']+K2+[R'][S]}=\frac{V_{max}[S][R]^n}{K_1[R]^n+K2+[R]^n[S]}
		\end{align*}

		\item  Confirm that when regulator and substrate are at very low concentration, the rate law in part (b) can be approximated as
		\begin{gather*}
			v=\frac{V_{max}}{K_2}sr^n.
		\end{gather*}

		When regulator and substrate are at very low concentration, $K_2\gg K_1r^n$ and $K_2\gg r^ns$, so
		\begin{gather*}
			v=\frac{V_{max}sr^n}{K_1r^n+K2+r^ns}\approx\frac{V_{max}}{K_2}sr^n
		\end{gather*}
		Then, we have already confirmed the argument.

	\end{enumerate}
\end{document}